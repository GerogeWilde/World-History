\section{古代罗马}
\subsection{罗马共和国的兴起}
公元前8世纪,在意大利的台河河畔,罗马城逐步建立起来。公元前509年罗马建立了共和国。

罗马共和国的兴起:罗马城源于台伯河河畔的若干部落的联合,即所谓的“七丘同盟”。

罗马早期共和国制度包括:
\begin{itemize}
    \item 执政官:均为贵族担任,正常有两个执政官,当有战事发生时,其中一个可被推选为独裁者;
    \item 元老院:由贵族组成,掌握国家的财政和军事,同时拥有对百人团大会决议的通过权,实际上拥有了国家的最高权力(早期贵族垄断着立法和司法大权,当时罗马只有习惯法,法律与习惯之间没有明显的界限。常常损害平民利益);
    \item 百人团大会:其中元老院贵族终身任职。“法西斯”象征着国家的最高权威。平民与贵族的斗争贯穿于罗马早期共和国历史。
\end{itemize}

(法西斯:古代罗马共和国时期的最高行政长官,有12个卫士相随。他们各手持一束棍棒,束棒中间插有一把战斧,象征着罗马国家的最高权利。棒子用于执行笞刑,斧子用来执行死刑。“束棒”在古代罗马人使用的拉丁语中,读作“法西斯”(fosces)。法西斯主义一词由此发展而来。

现代法西斯专政,是垄断资产阶级对内实行的极端专制的恐怖统治。)

方式:1.从战场上撤离;2.保民官(由平民推选),对贵族元老院的决策行使否决权,保障平民利益;

“十二铜表法”(贵族妥协后落实)。由于这些条纹镌刻在12块铜板上发表,故名。它涉及公法与私法,刑法与民法,同态复仇法,氏族继承与遗嘱等。

意义:是罗马第一部成文法典,传统的习惯法汇编而成,按律量刑,有法可依,对罗马法体系乃至近代欧洲都产生深远影响。

\subsection{罗马的扩张}

背景:经过平民与贵族的不断抗争,最终平民取得了较实质性的保障,国家趋于稳定,国力强盛。

公元前3世纪早期,罗马征服并统一了意大利半岛,然后向地中海地区扩张。公元前27年,罗马帝国建立,至1世纪后期,罗马帝国已经建立了30多个海外行省,控制欧亚非三大洲的广阔疆域,统治了许多不同的民族。

路线:1.先征服了意大利半岛;2.经过了三次“布匿战争”征服了西地中海;3.经过三次马其顿战争征服了东地中海。

主要原因在于罗马拥有优越的军事组织,在继承希腊的军事战略的同时又自我发展,形成了以军团为编制,军纪严格的军队。

对西地中海的征服中,由于罗马人称迦太基人为布匿人,史称布匿战争,持续了一百多年,可以分为三次较大的战争。

在第二次布匿战争(公元前218~201),迦太基(曾是腓尼基人在北非建立的殖民地)中诞生了一位卓越的军事将领汉尼拔,率军侵入意大利,征战15年不败。在最后一场战役中(公元前149年),罗马军队战胜了汉尼拔,迦太基头像(公元前3世纪末打败,迫使迦太基放弃了非洲以外的所有土地,并向罗马交付了大量赔款和几乎全部的舰只)。由于罗马人担心迦太基发展国力后东山再起,于是借机发动第三次布匿战争,彻底征服了迦太基,将其变为自己的一个行省。

经过三次马其顿战争,征服了东地中海。从此罗马彻底统治了地中海地区,将地中海变为自己的“内海”,清除了海盗,大力发展海上贸易,国力强盛。

罗马扩张后果:
\begin{itemize}
    \item 奴隶制的发展:奴隶源源不断地涌来,同时不被视为人看待;
    \item 大土地所有制的发展:服兵役的农民回家后穷困潦倒,不得不将自己的土地卖给贵族,贵族因此大肆兼并小农的土地;小农的破产导致罗马兵源的枯竭,罗马的公民制走向瓦解,而且罗马兵源的枯竭导致罗马共和国后期的冲突不断;
    \item 在征服过程中,不同民族之间的矛盾显现出来,特别是被征服者,由于得不到公民法的保护,对罗马统治表现出强烈不满。随着版图的扩展,国际交往的扩大,商品经济与贸易的发展,在政治经济活动中也产生了许多新问题,新矛盾。显然,仅适用于罗马内部的公民法已无法应对这些新变化;
\end{itemize}

为巩固统治,帝国对行省上层阶级大量授予公民权,对无公民法的外邦人给以适当的司法保障。3世纪,帝国境内自由民内部公民的区别不复存在,万民法成为适用于罗马统治范围内的一切自由民的法律。

\subsection{罗马共和国的危机与衰亡}

公元前1世纪,罗马发生了严重的社会危机。共和制再也无力统治,奴隶主企图建立独裁统治,以稳固政权。由于奴隶与奴隶主的冲突不断,因而奴隶们不断地爆发起义,其中最大的为斯巴达克斯起义(公元前73~71年),苏拉建立军事独裁(公元前49年)后,凯撒登上了历史舞台。

关于凯撒的三点:1.与庞培,克拉苏结成“前三头同盟”;2.远征高卢;3.托勒密王朝,埃及艳后;

在克拉苏死后,与庞培决裂,最后将罗马权利揽于其身,建立了独裁统治,结束了贵族的共和统治。贵族由于自身权利受到侵犯,于是刺杀了凯撒。之后其子屋大维崛起(公元前63~14年),与安东尼,李必达结成“后三头”同盟。并崛起后各自划分军政范围。屋大维在公元前27年开始独揽国家大权。屋大维最终与安东尼决裂,屋大维在打败安东尼后征服了埃及,托勒密王朝结束。

\subsection{罗马帝国的建立}

罗马帝制的建立(由屋大维建立,避开“独裁者”的身份,支持元老院的存在,但已名存实亡)。

“第一公民”,“首席元老”或“元首”;“奥古斯都”(拉丁文意为至尊或神圣);罗马帝制正式创立(屋大维成为第一位皇帝);“元首政治”是一种隐蔽的君主制(屋大维表面上维持共和国的框架,尽量尊重元老院地位)

屋大维为稳定帝国采取的措施:
\begin{enumerate}
    \item 改组军队,忠于自己;
    \item 消除社会动乱因素,通过“面包和竞技场”(来稳定社会上的无业游民);
    \item 推进罗马城市建设;
    \item 争取意大利同盟者的支持;
    \item 缓和对行省的政策(取消以前对各个行省的剥削)
\end{enumerate}

屋大维死后,立每年的8月为“奥古斯都日”,“August”

罗马摆脱了共和国晚期的动乱局面,经历了三个典型王朝(公元前14~公元前180年)。罗马帝国在最初的大约200年里,呈现出繁荣景象,是当时最强大的帝国之一。

屋大维统治罗马后,罗马多次发动侵略战争,帝国疆域不断扩大。到2世纪,它的疆域达到最大规模,东起幼发拉底河上游,西临大西洋,南抵非洲撒哈拉大沙漠,北达不列颠,莱茵河和多瑙河。

各个王朝

\begin{enumerate}
    \item 朱里亚·克劳狄王朝(公元前27年-公元68年)
    
尼禄是西方历史上一位著名的暴君,加剧了各个行省的矛盾冲突。 

\item 弗拉维王朝(69年-96年)

镇压犹太人起义,派重兵镇压耶路撒冷的犹太人起义。“犹太人”离散时期。

\item 五贤帝统治时期(公元前96~180年)

通过过继制度继承王位,是中央政权最稳定时期,即“黄金时代”。
    
\end{enumerate}

\subsection{基督教的产生及其早期发展}

\subsubsection{基督教的产生}
产生:公元1世纪20~30年代

基督教于1世纪诞生。起初,基督教信徒组成许多小宗教团体,后来这些小团体逐渐形成统一的基督教会。4世纪,罗马皇帝确定基督教为国教,大大促进了基督教的传播。后来,基督教逐渐传遍全欧。11世纪,基督教分裂为天主教和东正教,分别以罗马和君士坦丁堡为中心。

基本观点从犹太教中诞生,进而发展为普世宗教。

圣保罗对基督教的贡献:1.耶稣就是救世主;2.将基督教变为一种普世宗教。

《圣经》包含《旧约》和《新约》,《旧约》指上帝通过摩西与犹太教所定之约,而《新约》指上帝通过耶稣与信者另立之约。早期基督教徒受罗马当局迫害。

\subsubsection{基督教的发展}

基督教合法化的相关背景:
\begin{enumerate}
    \item 传播范围的扩大,信众人数增加,社会势力上升;
    \item 社会构成的变化
    \item 教义的变化:从主张平等,博爱,财产共有,互助合作到倡导劝人驯服,爱仇如己,忍受现实苦难,即从反抗的倾向转变为容忍的色彩。
\end{enumerate}

\subsubsection{基督教合法化}

过程:公元313年君士坦丁堡一世颁布“宽容赦令”——米兰赦令,承认基督教的合法地位。公元392年,皇帝狄奥多西一世正式宣布基督教为罗马的国教。至此,基督教才与帝国的政权相结合。

\subsection{罗马帝国的衰亡}

罗马帝国的成就(总人口达7000万以上)

维持了广大地区的秩序和稳定;普及公民权于各个行省,基本消除了征服者于被征服者的矛盾;建立了罗马共同体;建立了罗马公路网,交通四通八达;海上贸易繁荣 

原因:五贤帝时期最后一位皇帝没有将过继延续,最后酿成了罗马帝国时期“三世纪的危机”。

公元235~285最动荡时期:政局动荡和离心的倾向,和对王位的争夺,以及蛮族的威胁,日耳曼人的入侵,内部奴隶起义,基本造成了罗马帝国的崩溃,但由于公元3世纪末和4世纪初戴克里先和君士坦丁改革,维持了一段稳定。

主要内容:
\begin{enumerate}
    \item 公开的帝制——公开宣布君主制;
    \item 干预经济生活(将一定数量的土地和农民捆绑起来);
    \item 把帝国分为东西两部分,分别两位君主统治(395年);
    \item 加强宗教;
\end{enumerate}

君士坦丁则建立起东罗马帝国的新首都君士坦丁堡,在短期内加强了帝国统治,使罗马帝国度过了世纪危机。但于公元4~5世纪,罗马帝国崩溃。

蛮族入侵:

日耳曼人进入罗马帝国境内(由于受匈奴人西进驱赶);410年,日耳曼人分支之一西哥特人首次攻陷罗马;455年,汪达尔人再次攻陷罗马;476年,西罗马帝国灭亡(西欧封建社会的历史随之终结,东罗马帝国继续存在到15世纪中期)

\subsection{古代罗马的文化}

\subsubsection{宗教与文化}

罗马文化初期注重对希腊文化的传承,并且在希腊文化中融入了他们务实的风格。

宗教:古代罗马发展起的多神崇拜,深受希腊影响;

主神:朱庇特——宙斯;天后:朱诺——赫拉;智慧女神米涅瓦——雅典娜;战神:马尔斯——阿瑞斯;大力神:赫丘利斯——赫拉克里斯;美神:维也纳——阿芙罗狄黛;商旅之神:墨丘利——赫尔梅斯;胜利女神:维多利亚——尼凯

金星:维纳斯;木星:朱庇特;水星:墨丘利;火星:马尔斯;土星:萨图恩(农神)

希腊宗教特点:展示人性(人与神同性同形),而罗马的宗教则主要为政治性。

\subsubsection{语言与文字}

罗马使用拉丁字母,为以后欧洲各天主教文字作铺垫。

文学:维吉尔的史诗,贺拉斯的田园诗,奥维德的情诗。

建筑:古代罗马建筑,继承了希腊的建筑成就,包括穹顶,拱券。其建筑之所以更辉煌,在于将火山灰投入建筑使用中,加上沙,石头,基本等同于混凝土,方便按模具灌注成型。