\section{古代希腊}

\subsection{爱琴海文明}
公元前2000年左右,希腊早期文明——爱琴文明发祥于克里特岛,后来文明中心又转移到希腊半岛,出现迈锡尼文明。克里特文明和迈锡尼文明合称为爱琴文明。爱琴文明历时约800年后消亡。克里特岛是地中海最大的岛,海岸线曲折,东临爱琴海。

早——爱琴海文明:集中于克里特岛和东大陆伯罗奔尼撒半岛的迈锡尼;

晚——克里特文明:特点:1.中央集权;2.浓厚的宗教色彩;

\subsection{希腊文明的兴起}
希腊文化——荷马史诗:《伊利亚特》+《奥德赛》

\subsection{希腊城邦的建立}
\subsubsection{希腊城邦形成时期}
时间:(公元前8世纪~6世纪)

社会经济文化的进步:出现两百多个小国,史称“城邦”或“城市国家”。城邦面积狭小,人口不多,一般以城市为单位,包括周边若干村落。小国寡民和独立自主构成了城邦的基本特点;

铁器的普遍使用:青铜到铁(铁器的使用提高了生产力,开拓山峦,伐树造船,产生了剩余人口);

海外殖民(在地中海沿岸,围绕爱琴海),与母邦没有从属关系,在经济上联系密切。

文字重新出现(吸收了腓尼基文字的特点,同时首开从左往右写)。

\subsubsection{希腊城邦建立时期}
特点:以城邦为中心,连带周围农村地区(一城一邦,原意“堡垒”,城市与乡村的组合);

希腊早期(公元前8~6世纪)特点:小国寡民(最大8000多平方公里,几十万人口)。

标志城市:斯巴达与雅典。

\textbf{斯巴达(位于伯罗奔尼撒半岛南部)}

特点:并没有沿着雅典城市特有的民主政治发展,而是走向了寡头专制的政治体制,公民生活呈军事化;以农业立国,很少向外移民,而是以征服,扩张的形式,使得国家拥有了大量奴隶,奴隶和土地同属于国家,奴隶人口众多,约十倍于公民人口,因此斯巴达采取了军事化管理,强化自身的军事力量。

表现:1.优生;2.军事化的公民生活;3.“斯巴达式的教育”;4.妇女地位自由,重视妇女教育;

\textbf{雅典(建国,传说忒休斯建国)}

\begin{enumerate}
    \item 忒休斯改革
    
特点:雅典早期贵族统治(贫富分化严重,无力偿还债务者,变为奴隶);
    \item 梭伦改革

目的:为了全社会的利益(梭伦:古希腊七贤之一)

内容:颁布“解负令”(取消债务奴隶,解除了对于公民的威胁,扩大了公民的基础),不以出身而已财产等级,打破了贵族依靠门第垄断政权的局面,而非贵族出身的工商业者,奴隶主提供了参政,仪政的权利;

根据财产多寡,把公民分为四个等级,财产越多者等级越高,权力越大;恢复公民大会(雅典最高权力机构),设立400人会议(分为4个部落,每个部落100人,前三等级均可入选),打破了贵族垄断;

经济改革:设立新的,更受欢迎的陪审法庭来削弱贵族最高法院的权利;每个部落各选一将军组成十将军委员会;

意义:民主政治初现端倪,动摇了旧氏族的贵族特权,保证了公民的民主权利,为雅典民主政治奠定了基础。

    \item 克里斯特尼改革  
    
时间:公元前508年

内容:设立十个区域性部落组织,取代传统的四个血缘部落(按地区划分而不是基于氏族血缘关系,大大削弱了贵族的政治权利);

设立500人会议,由各部落轮流执政,吸收全体公民参政议政(没有财产限制,更加民主)。作用:为公民会议准备议案,同时具有最高执政权和行政权;

“陶片放逐法”作为民主政治的保障;

意义:到公元前500年,雅典已经出现民主政治,雅典民主政治进入成熟阶段。基本铲除了旧氏族的政治特权,公民参政权空前扩大,雅典民主政治确立起来。

\end{enumerate}

\subsection{城邦民主制的建立}

时间:公元前5~4世纪

背景:历时半世纪的希波战争(在亚洲西部波斯帝国不断扩展,侵犯到小亚细亚地区的城邦)

标志战役:
\begin{itemize}
    \item 马拉松战役(公元前490年,地点:马拉松平原),雅典战胜波斯;
    \item 温泉关战役(公元前480年)斯巴达战败,波斯长驱直入;
    \item 萨拉米海战(公元前480年)雅典战胜,波斯返回亚洲;
\end{itemize}

影响:从此世界格局并立发展;希波战争雅典取得盟主地位,民主政治达到全盛;

\textbf{“伯利克里时代”(公元前461~429)}

背景:公元前6世纪,古代伊朗以波斯人为中心形成了波斯帝国,波斯帝国频繁地出征和扩张,先后征服埃及等国际和地区。

公元前5世纪后半期,雅典人口约30万,其中几乎一半是外邦人和奴隶,成年男性公民只有5万。只有六分之一享受“民主”。雅典民主是建立在奴隶制度之上。

\textbf{与现代民主政治的不同}
\begin{enumerate}
    \item 直接民主(而非代议制的间接民主);
    \item 政府官员权利受到限制(任期短,一生只能从事两次);
    \item 全体成年男性公民可以参加最高权利机构公民大会,决定内政,外交,和平,战争等重大问题,他们在行政和司法机构中也发挥着重要作用;
    \item 业余人员组成的政府(有津贴,无专职官员,实行薪给制,贫民有可能担任公职)
\end{enumerate}

缺陷:成年男性公民的民主,无外邦人,妇女,奴隶;

补充:五百人议事会的职能也进一步扩大。陪审法庭成为最高司法与监察机关。法官从各部落30岁以上的男性公民中产生。他们审理各类重要案件,监督公职人员,并参加立法。同时建立许多由陪审团作最后决定的民众法庭,陪审员抽签产生,所有公民都可担任;

为鼓励公民积极参政,向担任公职和参加政治活动的公民发放工资。为吸引公民观赏戏剧,还特意为公民发放“观剧津贴”;

\subsection{古代希腊的文化}

(1) 神话

希腊人的宗教(神与人形象相同,性情也相似)

由此产生了戏剧,古典希腊时代希腊戏剧由政府主持,并且给观剧者发“观剧津贴”;

(2) 建筑

也由希腊宗教传说发展而来,包括雅典帕特农神庙,是为了供奉雅典娜女神而建。

材料由大理石而建,同时神庙的功能,不仅是城邦公民生活和商业活动的场所,而且还储藏着公共财富。

特点:1.方顶柱式结构;2.长方形的神庙殿堂;3.四周高大的圆柱柱廊;

(3) 哲学

希腊三贤:苏格拉底,柏拉图,亚里士多德

苏格拉底:拥护贵族统治,反对当时实行的城邦民主制;

柏拉图:哲学思想“理念论”;政治思想学说;“理想国”,同时在雅典城邦建立学园;

亚里士多德:“百科全书式”的学者

(4) 历史

希罗多德(西方史学之父)——《历史》(描述希波战争,歌颂雅典)

修昔底德《伯罗奔尼撒战争》(描述雅典与斯巴达之战,雅典战败)主张从客观,批判的角度看待历史。

\textbf{总结}

古代希腊是欧洲戏剧的故乡,涌现出许多著名的戏剧家。除索福克勒斯外,还有著名的“悲剧之父”埃斯库罗斯,以及“喜剧之父”阿里斯托芬。埃斯库罗斯的作品大都通过神话题材表现雅典政治制度和伦理道德的斗争,语言雄浑有力,寓意深刻。阿里斯托芬则用喜剧讽刺当时的政治,宗教和伦理道德。

雅典民主的理论与实践,为近现代西方政治制度奠定了最初的基础。民主氛围创造的空间,使雅典在精神文化领域取得辉煌成就。

雅典民主更是小国寡民的产物。过于泛滥的直接民主,成为政治腐败,社会动乱的隐患。选举和轮流执政,不能保证参议员的素质,国家权力有时成为社会不公的一种暴力机器。狭隘的城邦体制,最终无法容纳政治和经济的迅速发展。公元前4世纪后半期,日渐衰微的希腊被北部崛起的马其顿王国所灭。

实质:希腊民主是建立在奴隶制基础上的少数人的民主,维护的奴隶主贵族的利益。

\subsection{亚历山大帝国}

公元前4世纪下半叶起,希腊北部的马其顿崛起。

国王腓力二世改革,于公元前338年征服了希腊城邦,遭遇刺杀后,其子亚历山大大帝继承王位,征服了希腊,并向东扩张,打败了波斯帝国,最后其疆域地跨亚欧非三大陆,如图\ref{fig:亚历山大帝国}所示。

亚历山大帝国开启了人类历史上的新纪元,将东西方文明连接起来,最后定都巴比伦。政治措施:1、入乡随俗;2、联姻政策;3、移民


亚历山大去世后帝国一分为三:1、马其顿王国(希腊部分);2、埃及的托勒密(维持时间最长);3、塞琉西亚王国(以叙利亚为中心);

\subsection{希腊化时代的文化}

亚历山大东征后帝国重心东移,如埃及的亚历山大城。期间科学迅速发展,原因:1、希腊城邦时代的思想基础;2、近东的资料;3、王国政府的支持;

\subsection{古希腊乌托邦思想}
