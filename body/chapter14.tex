\section{第二次世界大战}
\subsection{绥靖政策}
面对法西斯势力的不断扩张,欧洲国家并未遏止法西斯势力,反而采取牺牲小国利益的方式来保全自身。最大的表现在以英法等大国操纵的国际联盟并未采取行动,在客观上鼓励了法西斯势力。

美国对日侵华举措采取不承认做法,以牺牲中国东北以维护美在华利益,不援助中国,鼓励了日本侵略。同时意大利对其它国家的侵略,国联不完全禁止。德国尝试入驻莱茵兰非军事区,法国只采取抗议而非军事行动。英法持续采取不干涉政策,美国采取中立。德国吞并奥地利,未遭到阻止。后英法等国签订《慕尼黑协定》,割让土地给德国,以换取本国内平。但此举进一步放纵了希特勒,并且巩固了其统治地位,并且也促进了德国经济,加速了世界大战的爆发。

\subsection{慕尼黑阴谋}
希特勒上台后,德国大肆扩军备战,还吞并了奥地利,却没有受到国家社会有效地抵制与制裁。德国的侵略野心进一步膨胀,又把侵略矛头指向捷克斯洛伐克的利益,求的一时太平。1938年9月,德意英法四国政府的首脑在德国的慕尼黑签订协定,规定捷克斯洛伐克必须在10天之内把苏台德等地割让给德国。历史上称这一事件为慕尼黑阴谋。

英国首相张伯伦对英国人说:“这是我们时代的和平,我建议你们安心睡觉吧!”

20世纪30年代,德日意法西斯在全国各地不断进行侵略活动,这些侵略活动使西方大国的利益受到一定损害。西方大国对法西斯国家的侵略不满,但又害怕法西斯国家的战争讹诈。他们想把祸水东引,把德的侵略势头引向苏联,因此,它们对法西斯侵略不是加以严厉制裁,而是希望以牺牲弱小国家的利益,来安抚侵略者。人们把这种政策成为绥靖政策,慕尼黑阴谋把绥靖政策推向顶峰。绥靖政策的影响极其恶劣,它使法西斯国家得寸进尺,侵略野心日益膨胀,也极大削弱了发法西斯力量。

德国入侵波兰后,英法开始采取强硬措施,并寻求与苏联合作。而苏联则与德国签订《苏德互不侵犯条约》,从而避免了德国的直接攻击,从而摆脱了两线受困的局面。而英法联盟的犹豫错过了组织法西斯的最佳时机。
20世纪30年代后半期,德意日三个法西斯国家相互勾结起来,结成了侵略性的军事政治集团,这三个集团成为“柏林-罗马-东京轴心”,又称轴心国集团。

\subsection{世界大战的爆发}
战争一开始,德军飞机就炸毁了波兰的机场,通讯枢纽等重要的战略设施。接着,两千多辆德军坦克冲垮了波兰军队的防线,这就是所谓的“闪电战”。英国和法国虽然宣战,却没有对德军发动进攻。面对德军,波兰军队进行了抵抗,挡住德军的进攻。这是,苏联有派兵占领了波兰的首都。不久,波兰覆亡。

1940年45月分,德军又向欧洲西部发动了大规模进攻,很快占领了包括法国在内的西欧和北欧许多国家,英国也遭到德军飞机的猛烈轰炸。在此期间,意大利也加入德国一方参战。

日本发动全面侵华战争以及德国突袭波兰,正式揭开了世界大战的序幕。英法并未立即出兵,按兵不动,“静坐战争”。而苏联为寻求自保,建立防线。后德国用闪电战攻占丹麦,挪威。1940年5月~6月入侵西欧各国,包围了英法军队。由于希特勒的犹豫,英法立即组织敦列尔克大撤退,摆脱了德国包围。后德军全面入侵法国,巴黎被占领。

法德签署停战协定,于贡比涅森林,接受条件解除全部武装,至此法兰西第三共和国灭亡。法国沦陷后,民主主义者戴高乐将军主张防抗,并积极投入战斗。

德国又攻击英国,丘吉尔首相坚决反对投降,抵住了德军的轰炸后,与德国空军发生“不列颠空战”,重创德国空军。

德国撕毁苏德和约,开始入侵苏联。1941年6月,德军掉头向东,发动了对苏联的侵略战争。不到几个月时间,德军便占领了苏联的大片国土,执笔苏联首都莫斯科。“巴巴罗萨”计划,闪电战。苏联严寒的冬天阻碍了德军,德军改全面进攻为重点进攻,后发生了“莫斯科保卫战”,斯大林率领顽强抵抗,德军战败,闪电战破产。

\subsection{美国的参战}
美国保持鼓励状态,后建立《新中立法》,向战争国出售武器。

罗斯福签署《租借法案》(1941年3月11日),对美国安全有重大影响的国家,美国可以进行租借,交换,借代等形式进行武器援助,由孤立主义转为参战倾向成为反法西斯国家的兵工厂,并且给苏联提供了军资。

罗斯福,丘吉尔签订了《大西洋宪章》。

日本在太平洋的不断扩张侵犯了美国利益,美国开始牵制日本,日本于1941年12月7日凌晨,日本军队偷袭美国太平洋基地珍珠港,第二天美国对日宣战。美国孤立情绪一扫而光,正式对日宣战,包括一系列太平洋国家。至此,太平洋战争爆发,成为名副其实的世界大战。

\subsection{世界反法西斯同盟的确立}
主要由社会主义国家,资本主义国家,半殖民地半封建国家组成

英美先结成同盟(卡萨布兰卡会议1943年1月14日~23日) 制定战略计划,后中英美三国举行开罗会议(1943年11月22日~26日),就战后对日本的处理举行会谈,美英苏德黑兰会议。

同盟为应急之用,内部有冲突,在共同利益之外存在诸多个人利益。

1942年1月1日,美苏英中等26个国家的代表在美国首都华盛顿举行会议,会议期间,各国签署了《联合国家宣言》,保证将用自己的全部人力物力,彻底打垮法西斯国家。

\subsection{欧洲、北非战场}
斯大林格勒战役(1942.7~1943.2)成为整个第二次世界大战的转折点,击败德军,使苏德战线西移,削弱了德军力量,使其转为防御。

阿拉曼战役(1942.10~11)是整个北非战场的转折点。意大利投降,并退出法西斯集团,对德宣战。

1944年6月6日,开辟第二战场,诺曼底登录(巴黎解放1944年8月)。

1945年4月攻入柏林,5月8日,德国正式签署投降书。

1945年2月,美国,英国,苏联三国首脑罗斯福,丘吉尔,斯大林在苏联的雅尔塔召开会议。会议决定打败德国之后,要对德国实行军事占领,彻底消灭德国的法西斯主义,同时,还决定成立联合国。苏联承诺在德国投降后三个月内,参与对日本法西斯的作战。

\subsection{亚洲、太平洋战场}
卢沟桥事变日本,日本发动全面侵华战争。开始时日本处于优势,几乎占领了大部分太平洋地区。转折点:中途岛海战(之后日本再也无力发动大规模战争)中国驻印度军及远征军逐步收复印度地区。

苏美英波斯坦会议后,9月2日,日本在密苏里战舰宣布无条件投降。最终结束了第二次世界大战,彻底摧毁了法西斯势力,人们获得和平。并且大大加速了欧洲的衰弱,国际政治格局发生巨达变革,由大国主宰的国际联盟时期转为美苏对峙的两级格局。

\subsection{联合国的确立}
美国提出的联合国方案为主,孜巴顿橡树园会议(1944年8月)
就几点存在分歧(1.创始成员国的身份;2.关于国家法庭的问题;3.殖民地托付问题)

雅尔塔会议解决了问题。1945年4月联合国成立。

