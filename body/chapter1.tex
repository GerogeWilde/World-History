\pagenumbering{arabic}
\section{人类的史前时代}
\subsection{人类起源}
地球发展的初期阶段时60亿年之前,46亿年前形成了地壳。从那时起,地球历史分为五个代:
\begin{enumerate}
    \item 太古代(约46亿年~25亿年)
    \item 元古代(约25亿年~6亿年)
    \item 古生代(约6亿年~2.25亿年)
    \item 中生代(约2.25亿年~7000万年)
    \item 新生代(约7000万年~至今)
\end{enumerate}

每一代分为若干纪,每纪分为若干世。

人类的进化过程大约是从森林古猿到南方古猿,直立人,早期智人(尼安德特人)再到晚期智人。至此出现了现代人种。

\subsection{旧石器时代的采集狩猎者}
% 为什么称之为旧石器时代?
% 采集狩猎者是什么鬼?
% 这一时代的特点是什么?

旧石器时代大体可以分为三个时期,分别为早期,中期和晚期。早期相当于最早的人属和直立人阶段,此时人类已经开始学会使用火。中期相当于早期智人阶段。晚期相当于晚期智人阶段。

在旧石器时代,人类主要使用石器为工具,石器制作技术在不同阶段逐渐成熟,在旧石器时代早期人类已能用火。

旧石器时代的人类以采集天然产物为主,如果实,块根,昆虫灯,同时也进行狩猎活动。采集和狩猎成为这个时代主要的劳动分工。火的使用在很大程度上加快了人类适应自然的进程,人类使用火种来进行御寒,保护自己,进而扩大了自己的活动范围。在晚期还出现了骨针,说明人们已经开始缝制衣服御寒。

生产技术的进步和生存条件的改善,导致了人类活动范围的扩大,人类开始向美洲及澳洲迁徙。大多数历史学家同意,人类通过东北亚的白令海峡的大陆桥进入美洲,此时正处于冰河时期,海平面降低。

\subsection{新石器时代的农业革命}
% 什么时间,发生了什么事
% 什么是新石器时代啊?
% 为什么认为是农业革命?
% 起因是什么?

从旧石器时代到新石器时代存在一段过渡时期,约15000年,从这时起,人类的经济活动有了进一步的发展。

同时,在中世纪时代,全球气候和生态环境发生显著变化。冰川期过去,全球气候转暖。经济活动内容扩大,生产工具也发生巨大变革。

新石器时代,人类发明了农业和畜牧业,掀起一场农业革命。

% \begin{enumerate}
% \item 清晰的组织结构,你可以在屏幕左侧看到他们
% \item 便捷的自动补充功能,只要输入命令的一部分就能够完成撰写
% \item 合理的宏包查看方式,右键菜单中可以找到宏包的文档
% \item 贴心的实用工具,矩阵插入助手,表格编辑助手等
% \end{enumerate}

























