\section{第二次工业革命及其影响}

\subsection{第二次工业革命的发生}
19世纪70年代,电力作为新能源进入生产领域。由于发电机和电动机的发明和使用,电力的应用日益广泛。电力逐步取代了蒸汽,成为工厂机器的主要动力,人类历史进入了“电气时代”。德国与美国走在前列。在第二次工业革命的推动下,资本主义国家的生产力获得了突飞猛进的发展,生产领域出现了垄断和垄断组织,生产的社会化程度增强。随着资本主义经济的繁荣和垄断组织的出现,资本主义国家开始从自有资本主义向垄断资本主义即帝国主义阶段过渡。19世纪末20年代初,主要资本主义国家美,德,英,法,日,俄等,相继进入了帝国主义阶段。

这一时期,主要资本主义国家出现了垄断组织,在美国和德国尤为突出。少数大资本家通过兼并和联合的方式,组成集团,控制一个或几个部门商品的生产,价格和市场,赚取高额利润,把越来越多的社会财富控制在自己手里。这种集团被称为垄断组织。垄断组织的出现,在一定程度上,适应了生产力发展的要求。但随着实力的增长,垄断组织越来越多地干预和控制国家的政治,经济生活。垄断组织还借助国家的力量,推行帝国主义政策,积极向外侵略扩张,剥削殖民地半殖民地国家。在世界范围内造成“穷者越穷,富者越富”的局面。

\subsection{第二次工业革命的特点与影响}
特点:1.科学成为推动生产力发展的重要因素;2.两次工业革命的广度,深度及影响力有所不同;3.几乎同时发生在几个先进的资本主义国家;

同时,由第一次工业革命时发展轻工业转为发展重工业,炼钢工业,化工工业迅速发展;动力由汽力转变为内燃机驱动,提高了效率。并出现了用内燃机驱动的汽车,飞机;进入了电气时代,用电力驱动;信息传递方式出现了有线电话和无线电;

科学成为推动生产力的主要因素,科学家与工程师充当了主要角色,主要在新兴的资本主义国家如美国,德国中开展,并引起了社会生产关系的重大变革,确立了垄断资本主义这种新形式,同时工业生产也越来越集中;

影响:彻底改变了封建时代的生产方式,确立了以重工业为核心的资本主义大生产方式。加快了城市化进程,与工业化相伴而行。工业额迅速发展推动了原有城市的扩大,并且逐渐改变了城市的功能与面貌,城市取代了乡村成为国民经济的中心。工业化带来许多新问题。如工业污染和对自然环境的破坏,社会问题,包括贫富两极分化,落后的市政建设与管理,每况愈下的社会治安,以及对妇女与童工的不公正待遇。

\subsection{盛极而衰的英国}
盛极而衰的历史根源:

1.沉重的历史包袱,主要是由于第一次工业革命取得的成就与辉煌,使英国沉迷于“世界工厂”的光环,忽视了新涌动的科技潮流,工厂主惰于引进先进设备进行革新,对新技术不敏感,不重视;

2.新一代技术力量形成缓慢,英国政府对教育,科技投入过少,致使缺乏新的技术人才;

3.企业经营管理方式落后,忽视企业经营,经济人才多转入政府阶层。

两党政治的实行(托利党-保守党,辉格党-自由党)

进行大规模的社会改革1.文官制度改革;2.教育改革。加强教育普及;3.建立大学与学院,培养技术性人才;4.议会制度的改革;5.社会经济改革。

在“维多利亚”时代,英国是世界上最大的殖民地国家。英国的资本主义垄断是建立在对殖民地的扩张与剥削的基础上。殖民地源源不断的原材料以及广阔的殖民市场,促进了英国的工业垄断,并且不断地对殖民地进行资本输出。在金融业,英国银行不仅控制着本国和殖民地财富,还影响着世界金融,伦敦为国际金融中心。

\subsection{缓慢发展的法国}
法国确立了共和制后,为巩固政体,进行改革。同时,由于缺乏大的政党,党派林立,政局不稳,严重影响经济发展。(教育改革,核心是推行世俗教育)。同时将天主教会清出教育机构,取消天主教会对教育的管理权,取消神学等。

法国经济缓慢发展的原因:
\begin{itemize}
    \item 割地赔款:普法战争的赔款巨大,同时所割地区为法国工业重地;
    \item 农业的衰退:农业过于分散,农民贫困,同时受外国农产品影响,农业水平下降;
    \item 金融资本,高利贷资本的片面发展,削弱了工业资本,与英国大量发展工业资本不同,法国不愿讲资本投入本国工业发展,片面发展借贷资本,延缓了国内发展速度;
    \item 动荡不安的政局(法德关系紧张)
\end{itemize}

\subsection{崛起的德国}
德国崛起的原因:

1.德国统一的完成,创造了国内统一市场;2.重视教育科技的发展;3.注重采用,推广最新科技成果;4.稳健的均势外交政策营造了良好的国际环境。

德国运用法国的50亿赔款大力发展,重视科技教育的发展,推动职业教育发展;同时德国具有后发优势,开始便建设电气设备;新技术的开拓(电力:韦尔纳·西门子;化工;化肥,染料;钢铁:炼轧一体化);同时国家对经济的干预也促进了经济的发展。

在第二次工业革命浪潮中,工人运动也大规模开展。德国威廉二世逼迫俾斯麦离开宰相一位,改变俾斯麦的大陆政策,开始实行全球政策,即对外扩张争夺殖民地,同时大力发展海军,增加军事开支,制定“史蒂芬计划”。

\subsection{快速发展的美国}
美国快速发展的有利条件包括:

1.独立后又经过南北战争得到统一,政治上得到相对稳定条件;

2.远远不断从其他国家进入外国移民,为工业革命提供了劳动力与技术;

3.南北战争统一后得到了广阔的国内市场;

4.农业资本主义迅速发展,并且农业不断进行机械化,确保了农业的迅速发展,为美国工业化提供了持续的保障;

5.具有优越的自然地理条件;

6.创新,并且善于把握机遇;

第二次工业革命步入了“电的时代”,逐步带动了资本主义的全面变革,确立了工业强国地位。工业化迅速发展,城市人口迅速增长,垄断组织迅速发展,托拉斯成为重要的经济力量。同时,银行业也开始了垄断,工业金融相互渗透,垄断巨头对社会影响越来越大,形成了军事扩张的野心,并且形成了“海权论”,迅速发展海军实力,企图扩张,争夺殖民地。

进入电气化时代,工业化浪潮滚滚向前,加剧了各国的殖民掠夺,逐步发展为帝国主义。
